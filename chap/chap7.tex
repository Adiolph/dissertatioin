\chapter{总结和展望}
\label{chap:conclusion}

在博士的研究中,我们围绕着海铃中微子望远镜在模拟上的设计预研以及硬件实验开展了多项工作:(1)中微子望远镜的模拟框架开发;(2)海铃的性能分析;(3)优化海铃的设计;(4)海铃探路者实验。

在第三章中,我们介绍了我们自主构建的一系列全链路的中微子望远镜的去物理模拟程序:
\begin{enumerate}
    \item 针对中微子事件产生子,我们一同参与开发了由德国KIT的宇宙线小组主导研发的最新模拟高能粒子模拟程序框架——CORSIKA8,它拥有现代C++的设计和丰富的可拓展性。利用该框架,我们实现了对中微子与原子核DIS过程顶点的带权重的采样。我们在框架下集成了\textsf{Pythia8},\textsf{UrQMD},\textsf{Sibyll},\textsf{PROPOSAL}和\textsf{TAUOLA}等成熟的物理模块,来产生DIS反应的末态粒子,并它们的强子簇射和缪子径迹传播到探测器的过程,以及陶子衰变产生额外的簇射的过程。
    \item 对于望远镜内部阵列的粒子簇射模拟,我们先是采用\textsf{Geant4}来模拟簇射演化的过程,从而得到能够产生切伦科夫光子的带电粒子径迹。但在随后的开发中,为了解决逐个例子粒子模拟的计算资源消耗问题,我们使用了参数化的方式来代替簇射中电磁成分的模拟,从而得到了约1000倍的性能提升。
    \item 对于切伦科夫光子的传播模拟,我们开发了一套GPU加速的光线追踪模拟。该程序基于\textsf{NVIDIA-OptiX7}的光线追踪库,能够发挥GPU中的大型并行计算能力,并且利用GPU中的光锥核心来实现硬件层面的加速。我们的程序在最新的A100显卡上能够实现每秒$4\times10^8$切伦科夫光子传播的计算能力,是传统CPU光线追踪速度的1000倍。
    \item 为处理hDOM内部复杂的几何结构,我们在\textsf{Geant4}框架下对hDOM的内部几何进行了详细的建模,从而能够准确的模拟光子从hDOM表面一步步地传播到其中的PMT和SiPM过程。借助于\textsf{Geant4}中对光学界面的精细处理,我们可以探究hDOM内部结构设计的优化。
    \item 我们还设计了光敏元件对入射光子响应的模拟。该程序以光子打在PMT光阴极上的信息作为输入,根据实验室测量得到的PMT硬件性能,对PMT最终输出的波形信号进行模拟。
\end{enumerate}

在第四章中,我们基于上述开发的模拟系统,对海铃中微子望远镜的性能进行了分析。在目前的分析中,我们专注于由缪子中微子产生的径迹型事件。
\begin{enumerate}
    \item 我们模拟缪子径迹事件在中微子望远镜中的信号,根据阵列中hDOM内光敏元件的测量结果,来对缪子径迹进行了方向重建。重建中以光敏元件的位置和其测量到的光子到达时间为主要输入量,使用了最大似然估计的方法来寻找缪子径迹方向的最优解。我们发现海铃对$1\,\mathrm{TeV}$的缪子中微子可以达到略小于$1^\circ$,而对于$1\,\mathrm{PeV}$的事件则可以达到略小于$0.1^\circ$,实现了世界最顶尖的水准。
    \item 通过对缪子中微子事件进行完备的采样以及触发判断,我们得到了海铃对缪子中微子的有效面积,其在$1\,\mathrm{TeV}$可以达到$\sim 10^2 \mathrm{m^2}$,而在$1\,\mathrm{PeV}$可以达到$\sim 2 \times 10^3 \mathrm{m^2}$。这样的有效面积是目前IceCube的10多倍。
    \item 根据海铃的有效面积,我们可以针对中微子源计算其探测的事例率。而根据海铃对中微子的角分辨率,我们可以在空间上对中微子源进行显著性分析。以NGC 1068这个目前最有可能的中微子源为例,我们假设IceCube的最佳拟合流强并考虑大气中微子和弥散高能中微子作为背景,我们发现海铃凭借着其庞大的探测器体积和优秀的角分辨能力,能够在1年内便以$5\sigma$的置信度发现NGC 1068。并且由于海铃的低纬度的特性,它对全天各个赤纬的中微子源都有能够超越IceCube-Gen2的灵敏度。
\end{enumerate}

在第五章中,我们利用上述搭建起来的模拟和分析流程,对中微子望远镜中的一些设计进行了一定的优化。
\begin{enumerate}
    \item 我们探究了同时包含SiPM和PMT的hDOM对缪子径迹型事件的角度分辨率提升。在研究中,我们主要考虑了SiPM的快速时间响应能力和高量子效率的优势。我们发现相比于传统的只包含PMT的mDOM,hDOM在海水这样的低散射介质中能够得到$\sim 40\%$的角分辨率提升。这对于寻找微弱的中微子源而言是至关重要的。
    \item 我们提出了一种独特的中微子望远镜中串列单元的排布方式——彭罗斯构型。该构型借鉴于彭罗斯镶嵌——一种非周期性的空间密铺结构。我们发现该构型中串列单元的空间密度分布疏密有秩,因而有可能能够拓宽望远镜对簇射型事件的灵敏能量范围。此外该构型不具备明显的方向性,能够避免直接漏过探测器的时间,并对来自各个水平方向上的大气缪子事件进行有效的屏蔽。
    \item 我们对阵列中hDOM的水平和垂直间距的大小对望远镜性能的影响进行了分析,通过对一些列不同水平和垂直间距的阵列配置进行模拟和分析,我们发现垂直间距放大到1.2倍有利于探测能谱指数为-2的源;而将水平间距缩小到原来的0.7倍则有利于探测能谱指数为-3的中微子源。
\end{enumerate}

在第六章中,我们对探路者实验中的实验设计,PMT性能测试,PMT系统刻度,以及基于模拟的实验数据分析进行了回顾。
\begin{enumerate}
    \item 参考过去其他中微子望远镜中的海水测量实验,我们设计了由一个发光模块和两个接收光模块构成的对海水光学性质进行相对测量的实验装置。该实验装置中包含PMT和相机这两套独立的系统,可以各自对海水的性质进行测量。通过对南海水下$3400\,\mathrm{m}$的海水环境进行原位测量,我们得到此处海水的等效衰减长度约为$25\,\mathrm{m}$,与世界上其他深海中微子望远镜的测量结果接近。
    \item 我们基于Geant4开发了一套海水在光子下传播的过程的模拟程序,基于该程序,我们对实验的实验现象进行了一定的预测,对实验设计中的一些参数的确定以及后期实验数据分析方法提供了辅助指导依据。最后程序的模拟结果配合实验室刻度的结果被用于与真实海试测量得到的实验数据进行对比,通过最小$\Chi^2$误差的方式得到了海水中的光学性质。
    \item 我们完成了对PMT系统的整体设计和采数方案的设计,对实验所使用的PMT进行了完善的测试,对封装好的PMT系统在低温的环境下进行了整体的刻度实验。在探路者实验为我们对PMT的性能,电子学采数系统的设计,以及海洋工程的实施留下了宝贵的实战经验经验。
\end{enumerate}

海铃计划是一个新提出的项目,目前仍然处在比较早期的设计预研截断,其中还有许多内容需要继续完善。
\begin{enumerate}
    \item 在模拟方面,我们需要对模拟过程中涉及到的物理过程,特别的强子作用的部分进行更详细的物理检验。在保证一定模拟精度的前提下,我们需要进一步提高模拟的运行效率,这可以通过对一些物理过程进行参数化和查表的形式实现。我们还需要不断扩充模拟框架的功能,例如使其涵盖更大的能量范围,支持对超出标准模型的物理过程的模拟。不容忽略的是,我们还需要提高整个模拟框架的易用性,可拓展性和稳定性。以上的这些要求对海铃模拟团队的物理,数学和编程能力都是不小的考验。
    \item 在探测器性能分析方面,我们除了需要加大分析的统计量以外,还要考虑更多额外维度的问题,例如对目前我们只讨论了中微子能量对方向重建精度的影响,但实际上中微子的入射倾角也会对重建精度有一部分影响。在对探测器的有效面积分析和灵敏度分析中,我们的事件筛选标准还比较初步,并没有考虑更多的噪声的影响,例如海水中的K40放射性背景,错误重建的大气缪子等。
    \item 在探测器构型的设计上,我们还需要更多详细的研究,例如像IceCube + DeepCore\cite{IceCube_deepcore:2011}这种一个稀疏的大阵列中嵌入一个密集的小阵列,以及Baikal-GVD\cite{BAIKAL_design:1997}和P-One\cite{P-ONE:2020}这种分块构型的阵列,都是可以探讨的对象。此外我们还需要更多精细的模拟验证,来定量地分析各个阵列下的性能。
    \item 在hDOM的硬件实现方面,我们还需要对SiPM的硬件性能做更多的测试。由多个单点SiPM并合成的阵列存在时间响应性能劣化和暗噪声率过高的问题,这些都是需要硬件上去不断地测试和寻找解决办法的。
\end{enumerate}

海铃计划从正式提出到现在不过3年左右的时间,在如此短暂的时间内,我们已经完成了大量的预研模拟工作以及硬件测试工作,对海洋工程方面的设计和建设也有大量的调研和测试。
尽管前方还有不计其数的技术难题需要解决,但我们相信在大家的不懈努力下,中国南海的中微子望远镜一定能够最终建设成功,并且成为国际上领先的观测设备。