\chapter{致谢}

岁月匆匆,转眼间已经过去了六年,回想起当初在古风古雅的科维理研究所决定攻读天体物理博士的那个夏天,那时的我没有多少科研经验,只有对物理和天文的纯粹的向往。
读研究生后,我才深刻体会到科研的艰辛,科研工作不像是上课做题考试那样容易得到正反馈。
真正的科研大部分时候都是枯燥乏味的。在模拟和分析时,一行隐蔽的代码bug能让人挠头三天,写错一段物理表达式或者漏掉一个数据处理步骤就需要重做一周乃至一个月的工作。
而做动手实验的过程也难免有大量的重复性体力劳动,扑朔迷离的实验结果让人不禁怀疑:到底是仪器有问题还是人有问题?
每一个科研成果的背后都是由默默无闻的工作,激烈的组会讨论,以及详尽的合作组审核所组成的。
尽管如此,每当取得一点小小的成功后,那种喜悦感足以化解一周甚至一个月的忧愁。

在此,我要感谢所有在博士期间给予我研究和生活上帮助的人们,是在大家的帮助和共同努力下,我才能顺利完成博士研究。

首先我要感谢我的研究生导师黎卓老师。黎老师为人温和谦逊,与我曾经对科学家形象的想象完全一致。
在参加每周的高能小组组会时,我常常被黎老师对高能天体辐射机制这个领域的广泛而又深刻的理解所震撼,他带领我走进了高能中微子的领域。
尽管我后来没有选择延续做老师给的超高能中微子辐射的课题而是希望去做下一代中微子实验的预研,黎老师对此也非常地支持。
他不仅将我介绍给李所的徐东莲老师,还愿意让我在上海长期做研究。
在我前往交大做研究时,黎老师也没有因为个人研究兴趣或者出于经费和项目压力给我布置额外的任务,给予了我最大的研究自由。

其次,我要感谢徐东莲老师,徐老师是我在交大和李所的第二导师,是她指导我完成了大部分的博士工作。
徐老师作为一名刚回到国内的青年学者,却敢于去推动需要数十亿投入的中微子望远镜项目。
这个过程中拼尽心血,将自己的才能和精力都奉献给这一项重大科学基础项目的建设中。
徐老师让我感受到了对科研工作的热爱与执着,以及对理想的不懈追求。
是徐老师带领我们这样一个年轻的,甚至是研究生和本科生为主的团队,从零开始建设中国的中微子望远镜项目——海铃计划。
如果说科研像是西天取经,那么徐老师在团队中就像是唐僧一样的角色。
我们团队中的每一个人都有自己的擅长之处和不足之处,大家在科研和生活中也都遇到形形色色的困难,
是徐老师坚定的信念让我们能够克服一路上的艰难险阻,实现了“海铃速度”。

除了两位导师以外,我还要感谢北大徐仁新老师,他有非常好的物理直觉,对我的物理品味产生了深远的影响。
更重要的是徐仁新老师始终发自内心地为学生和中国的科研事业着想,希望学生得到最顶尖的培养。
他帮助我申请了去英国参加致密星和引力的暑期学校,让我作为一个非领域内小同行的人也能出去感受国际上的研究和学术氛围。
交大的刘江来老师不仅给我们海铃项目团队提供了战略支持,还经常切身参与到我们实验讨论中。
在遇到实验难题时,我们时常说的一句话便是“问问刘老师怎么看”。
田新亮老师是海铃海洋工程的灵魂人物,他是我接触到的第一个真正的工科老师。
田老师对工程问题的深远见解和超强的问题解决能力帮助海铃团队走到了今天,也让我了解到了工程科学的魅力。
在科研生活中,还有许多老师值得我的感谢和敬佩:
Kohei Inayoshi精彩绝伦的量纲分析给我对物理的理解打开了新世界;邵立晶老师严谨的科研态度和对教学的热爱也成为了我心中的标杆;
郭军老师在海铃团队的论文写作起了很大的帮助;谌勋老师是交大粒子所集群的管理员,他帮助我和其他人解决了非常多的稀奇古怪的程序问题;
高能所的何会海老师是一个非常纯粹的物理学家,与他有关中微子望远镜设计的讨论令我受益匪浅;
KIT的Felix Riehn指导我完成了多项CORSIKA8的改进工作等等。

我还要感谢北大天文系18级的同学们:高勇,焦文裕,徐德望,徐江伟,张中府,胡豪杰,李文秀,史芳菲,刘畅,以及何子麒和陆家明。
在研一时期与大家一同上课学习和生活的时间,我觉得我们像最紧密的伙伴,那个时候我们朝气蓬勃,对学术充满热情。
在食堂和路上,我们能聊一些上课时的问题,听完讲座后的感想,或者是听高勇说一些名人往事和学术八卦。
在宿舍内,我也经常和我的室友焦文裕以及楼上的徐德望一起畅谈科研与生活,
我们一起分享趣闻,吐槽不顺的事情,还可以联机打文明6。
有了你们的相伴,在北大的生活才有滋有味。

北大高能小组内的师兄师姐和师弟们也给予了我很大的帮助。
黄天奇师兄,王凯师兄,黄艳师姐和朱锦平师兄都为我解答过许许多多在天体物理模型,中微子产生等方面的问题。
田喜水和张秦源虽然是我的师弟,但我也从他们身上学习到了很多知识,我们讨论过许多有关数理统计和中微子辐射方面的问题。
在博士最后一年中,我们一起在办公室干到晚上近7点,再一起去家园三楼吃晚饭,也是非常有趣的时光。

交大海铃团队的同学们是我在上海两年多的时间内最精密的科研和生活伙伴。
叶子平师兄,张飞洋师兄和张淯翎师姐在交大给予了我很大的帮助。
田玮,李文莲,张符雨笛,孙正阳,王铭鑫,咸世莘,唐剑男,魏振宇,智威,常起超,薛峤,黄惊涛,莫岑,李家琳都是在海铃团队一起共同奋斗过的伙伴。
我们一起在实验室工作和生活,在周末时一块去龙湖天街和永平路聚餐,在疫情期间线上连麦闲聊。
在110实验室的生活构成了我一半的博士时光。
在海铃探路者实验期间,我们一起设计了实验方案和各个参数,在组装测试玻璃球时我们一起加班到深夜,在向阳红3号上我们轮班通宵作业。
再后来加入的博后Iwan Morton-Blake帮助我们指导了许多英语写作,他对海铃中的物理问题也有着非常清晰的解决思路,在生活上也经常请大家吃饭和游玩。
张宇雷和粟杨捷尽管没有加入海铃团队,但他们在海铃模拟框架的搭建上也帮助贡献了许多代码。

我还要感谢我在博士暑期实习期间遇到的网易次会预告工作室的领导刘轻舟和向先松,
他们平易近人,完全没有领导的架子,并且愿意赏识选择转行的我。
工作室的同事们,李昱钊,李林,徐立栋和甘雯伟也为我当时的工作和生活提供了很大的帮助。
他们对我的接待让我能够在最后一年心无旁骛地对科研工作进行整理和收尾。

我的朋友们,周家豪,胡建雄,敖乐敏,谢新宇也在我的生活上给予了很大的帮助。
人生难得能有几位同好和知己,能与他们聊游戏,聊动漫,分享生活,交流观点是非常难得的体验。

最后,我要感谢我的妻子和家人们。
我的妻子王瑶在我们恋爱后每天晚上都跟我保持一个小时的语音通话,
她会跟我讲述自己记者工作中见到的趣闻,在我家生活时发生的新鲜事,或是时政娱乐八卦。
她总是那么乐观开朗,像是一个小太阳,向外散发着正能量,刚好与我内向的性格互补。
能与她一起生活是我对未来最美好的憧憬。
我的父母虽然都是普通的农民出身,但他们对我的个人选择从来不加以干涉,给了我自由幸福的成长环境。
是他们的默默支持帮助我走到了今天。
